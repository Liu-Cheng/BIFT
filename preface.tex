%%%%%%%%%%%%%%%%%%%%%%preface.tex%%%%%%%%%%%%%%%%%%%%%%%%%%%%%%%%%%%%%%%%%
% sample preface
%
% Use this file as a template for your own input.
%
%%%%%%%%%%%%%%%%%%%%%%%% Springer %%%%%%%%%%%%%%%%%%%%%%%%%%

\preface
If your computer crashes, you can revive it by a reboot, an empirical solution that usually turns out to be effective. The rationale behind this solution is that transient faults, either in hardware or software, can be fixed by refreshing the machine state. Such a “silver bullet”, however, could be futile in the future because the faults, especially those existing in the hardware such as Integrated Circuit (IC) chips, cannot be eliminated by refreshing. What we need is a more sophisticated mechanism to steer the system back to the right track. The “magic cure” is the on-chip fault-tolerant mechanism, which relies on a suite of built-in design-for-reliability logic, including fault detection, fault diagnosis, and fault recovery, working in a unified manner. 

With the shrinking semiconductor feature sizes and continuously scaling of the IC designs, silicon defects caused by manufacture defects, radiation particles, or progressively aging are almost inevitable and pose critical influence on both the yield and quality of IC products. Under this context, we have successfully applied on-chip fault-tolerant computing mechanism onto a set of different chip designs including generic circuits, general purposed processors, network-on-chips, and deep learning processors in the past decade, and gradually formulate an systematic built-in fault-tolerant computing paradigm, which can be utilized to guide IC designs against these typical silicon defects. In addition to the basic fault detection, fault diagnosis, and fault recovery, the proposed built-in fault-tolerant computing paradigm also provides additional benefits, such as facilitating graceful performance degradation, mitigating the impact of verification blind spots, and improving the chip yield.

In this book, we mainly illustrate the built-in fault-tolerant computing paradigm with practical demonstrations on genetic circuits, general purposed processors, network-on-chips, and deep learning processors. The entire book consists of six chapters. Chapter 1 presents the background of fault-tolerant chip designs and overview of the built-in fault-tolerant computing paradigm. Chapter 2 presents on-line fault detection, on-chip path delay, and lifetime fault-tolerant pipeline design for genetic circuits. Chapter 3 investigates the vulnerability of general purposed processors under silicon defects and presented a core salvaging approach particularly for multi-core processor architecture. Chapter 4 focuses on fault-tolerant network-on-chip designs from distinct angels including topology reconfiguration, routing design, and architecture design. Chapter 5 focuses on built-in fault-tolerant deep learning processors fabricated with both conventional CMOS-based technology and emerging ReRAM-based technology. Chapter 6 concludes this book with a brief summary of the proposed built-in fault-tolerant computing paradigm and discussion of future fault-tolerant computing directions on large-scale VLSI designs.

The majority content involved in this book is collected from peer-reviewed papers of Guihai Yan, Cheng Liu, Lei Zhang, Wen Li, Songwei Pei, Bingzhang Fu, Ying Wang, and Hang Lu supervised by Prof. Xiaowei Li, and has already been published in the journals of TVLSI, TCAD, TC, JCST, and Journal of China Science. Prof. Xiaowei Li organized this book in general, Prof. Guihai Yan mainly worked on Chapter 2 and Chapter 3. Prof. Cheng Liu worked on Chapter 1, Chapter 4, Chapter 5, and Chapter 6. Dr. Jingya Wu helped edit this book. Prof. Huawei Li and Prof. Guojie Luo reviewed this book. Prof. Tim Cheng wrote foreword for this book. All the efforts are indispenable for this book and greatly appreciated.

The techniques presented in this book are partly selected from research founded by NSFC No.62174162 and No.61902375. 

\vspace{\baselineskip}
\begin{flushright}\noindent
\leftline{SKLCA, Institute of Computing Technology, Haidian, Beijing} \\
May 2022\hfill {\it Xiaowei Li}\\
\end{flushright}


